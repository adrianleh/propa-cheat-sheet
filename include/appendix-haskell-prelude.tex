% !TeX root = ../propa-cheat-sheet.tex
\appendix\section{Haskell Prelude functions}
\subsection{Collections}
\begin{minted}[breaklines]{haskell}
elem :: Eq a => a -> t a -> Bool 
maximum :: forall a. Ord a => t a -> a
minimum :: forall a. Ord a => t a -> a
sum :: Num a => t a -> a
\end{minted}
\subsubsection{Lists}
\begin{minted}[breaklines]{haskell}
map :: (a -> b) -> [a] -> [b]
(++) :: [a] -> [a] -> [a] -- Concat lists
filter :: (a -> Bool) -> [a] -> [a]
head :: [a] -> a -- first element
last :: [a] -> a -- last element
tail :: [a] -> [a] -- everything except the first element
init :: [a] -> [a] -- everything except the last element
null :: Foldable t => t a -> Bool -- Empty check
length :: Foldable t => t a -> Int
reverse :: [a] -> [a]
(!!) :: [a] -> Int -> a -- Index operator
any :: Foldable t => (a -> Bool) -> t a -> Bool -- Test wether any element satisfies a predicate
all :: Foldable t => (a -> Bool) -> t a -> Bool -- Test wether all elements satisfies a predicate
replicate :: Int -> a -> [a] -- replicate n x is a list of length n with x the value of every element.
\end{minted}
\subsubsection{Infinite lists}
\begin{minted}[breaklines]{haskell}
iterate :: (a -> a) -> a -> [a] -- iterate f x == [x, f x, f (f x), ...]
repeat :: a -> [a] -- infinite list from first element: repeat x = [x, x, ...]
cycle :: [a] -> [a] -- turn finite list into circular infinite list
\end{minted}
\subsubsection{Sub-Lists}
\begin{minted}[breaklines]{haskell}
take :: Int -> [a] -> [a]
drop :: Int -> [a] -> [a]
splitAt :: Int -> [a] -> ([a], [a])
takeWhile :: (a -> Bool) -> [a] -> [a]
dropWhile :: (a -> Bool) -> [a] -> [a] 
break :: (a -> Bool) -> [a] -> ([a], [a]) -- break, applied to a predicate p and a list xs, returns a tuple where first element is longest prefix (possibly empty) of xs of elements that do not satisfy p and second element is the remainder of the list
zipWith :: (a -> b -> c) -> [a] -> [b] -> [c]
zip = zipWith (,)
unzip :: [(a, b)] -> ([a], [b])
\end{minted}
\subsubsection{Folding}
\textbf{Right to left:}
\begin{minted}[breaklines]{haskell}
foldr :: (a -> b -> b) -> b -> t a -> b
\end{minted}
Usage:
\begin{minted}[breaklines]{haskell}
folded = foldr (\val -> \acc -> newAcc) acc coll
\end{minted}
\textbf{Left to right:} 
\begin{minted}[breaklines]{haskell}
foldl :: (b -> a -> b) -> b -> t a -> b
\end{minted}
Usage:
\begin{minted}[breaklines]{haskell}
folded = foldl (\acc -> \val -> newAcc) acc coll
\end{minted}
\subsubsection{Tuples}
\begin{minted}[breaklines]{haskell}
fst :: (a,b) -> a
snd :: (a,b) -> b
curry :: ((a, b) -> c) -> a -> b -> c
uncurry :: (a -> b -> c) -> (a, b) -> c
\end{minted}
\subsection{Values and numbers}
\begin{minted}[breaklines]{haskell}
succ :: a -> a
pred :: a -> a
even :: Integral a => a -> Bool
odd :: Integral a => a -> Bool
gcd :: Integral a => a -> a -> a
lcm :: Integral a => a -> a -> a
(^) :: (Num a, Integral b) => a -> b -> a 
\end{minted}
\subsubsection{Float}
\begin{minted}[breaklines]{haskell}
isNaN :: a -> Bool
isInfinite :: a -> Bool
\end{minted}
\subsection{Misc}
\subsubsection{IO}
\begin{minted}[breaklines]{haskell}
putStrLn :: String -> IO ()
print :: Show a => a -> IO ()
getLine :: IO String
getChar :: IO Char
\end{minted}
\subsubsection{Maybe}
\begin{minted}[breaklines]{haskell}
data Maybe = Nothing |  Just a
\end{minted}
\begin{minted}[breaklines]{haskell}
maybe :: b -> (a -> b) -> Maybe a -> b
\end{minted}
Usage:
\begin{minted}[breaklines]{haskell}
value = maybe orElse mappingAToB maybe instance
\end{minted}
