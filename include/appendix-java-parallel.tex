% !TeX root = ../propa-cheat-sheet.tex
\appendix\section{Java Reference}
\paragraph{\texttt{java.util.Thread}}
\begin{minted}[breaklines]{java}
void Thread#run()
boolean Thread#isInterrupted()
void Thread#interrupt()
boolean Thread#isAlive()
void Thread#sleep(int)
void Thread#join()
\end{minted}
\paragraph{\texttt{volatile}}
\begin{minted}[breaklines]{java}
class A {
    volatile Type name; // prevent caching and reordering around usages
}
\end{minted}
\paragraph{\texttt{AtomicInteger}}
\begin{minted}[breaklines]{java}
AtomicInteger(int)
int AtomicInteger#get()
int AtomicInteger#incrementAndGet()
int AtomicInteger#decrementAndGet()
boolean AtomicInteger#compareAndSet(int, int) // first parameter oldVal, second parameter sndVal; replaces iff atomic integer has value oldVal
\end{minted}
\paragraph{\texttt{Locks}}
\begin{minted}[breaklines]{java}
void Lock#lock()
void Lock#unlock()
boolean Lock#tryLock() // only take lock if it's avb
\end{minted}
Implementation of interface: \texttt{ReentrantLock}
\begin{minted}[breaklines]{java}
ReentrantLock​(boolean) // Constructor with fair option
\end{minted}
\paragraph{\texttt{Semaphore}}
Allows n threads into a critical section
\begin{minted}[breaklines]{java}
Semaphore(int, boolean) // Constructor with amoint and fair option
void Semaphore#aquire()
void Semaphore#release()
boolean Semaphore#tryAquire()
\end{minted}
\paragraph{\texttt{CyclicBarrier}}
\begin{minted}[breaklines]{java}
CyclicBarrier(int)
void CyclicBarrier#await() // once called n times, all threads resume
\end{minted}
\paragraph{\texttt{CountDownLatch}}
Just like cyclic barrier only non-reusable: After n calls of \texttt{await} let's all threads through

\paragraph{\texttt{ExecutorServices}}
\begin{minted}[breaklines]{java}
Executors.newSingleThreadExecutor()
Executors.newFixedThreadPool(int)
Executors.newCachedThreadPool()
\end{minted}
Example:
\begin{minted}[breaklines]{java}
ExecutorService executor = Executors.newSingleThreadExecutor(); executor.execute(() -> {
    String threadName = Thread.currentThread().getName(); System.out.println("Hello " + threadName);
});

try {
    executor.shutdown();
    executor.awaitTermination(5, TimeUnit.SECONDS);
} catch(InterruptedException ex) { } finally {
     if (!executor.isTerminated()) {
          executor.shutdownNow();
     } 
}
\end{minted}
\paragraph{\texttt{Futures}}
\begin{minted}[breaklines]{java}
<T> Future<T>#submit(Callable<T> callable)
\end{minted}
 
\begin{minted}[breaklines]{java}
ExecutorService executorService = Executors.newCachedThreadPool(); 
List<Future<Integer>> futures = new ArrayList<Future<Integer>>(); 
for (int i = 0; i < 10; i++) {
    final int currentValue = i;
    futures.add(executorService.submit(() -> { return currentValue; }));
}
executorService.shutdown(); // Make sure all values are in
for (Future<Integer> future : futures) { 
    try {
         Integer result = future.get();
         System.out.println(result);
    } catch (ExecutionException ex) {}
} 
\end{minted}

\paragraph{\texttt{Completeable Future}}
\begin{minted}[breaklines]{java}
CompletableFuture<Integer> futureCount = CompletableFuture.supplyAsync( () -> {
    try {
         // simulate long running task 
         Thread.sleep(5000);
    } catch (InterruptedException ex) { }
    return 20; 
    }
);
CompletableFuture<String> modified = futureCount.thenApply((Integer count) -> {
     int transformedValue = count * 10;
     return transformedValue; 
     }
).thenApply(transformed -> "Finally create a string: " + transformed);
System.out.println(modified.get());
\end{minted}


\paragraph{\texttt{BlockingQueue}}
\begin{minted}[breaklines]{java}
T BlockingQueue<T>#take()
BlockingQueue<T>#put(T)
\end{minted}
Concrete impls:
\begin{itemize}
	\item LinkedBlockingQueue - unlimited
	\item ArrayBlockingQueue - limited
	\item PriorityBlockingQueue - sorted
\end{itemize}

\paragraph{\texttt{ForkJoin}}
\begin{minted}[breaklines]{java}
ForkJoinPool fjPool = new ForkJoinPool();
MyTask myTask = new MyTask(...);
fjPool.invoke(myTask);
\end{minted}
\begin{minted}[breaklines]{java}
public class MyTask<T> extends RecursiveTask<T> {
@Override
public T compute() {
     if(baseCase) { return baseValue; }
     // calculate something and return if problem small
     MyTask leftTask = new MyTask(...); 
     MyTask rightTask = new MyTask(...);
     leftTask.fork();
     T right = rightTask.compute();
     T left = leftTask.join();
     return operation(left, right);
}
}
\end{minted}
